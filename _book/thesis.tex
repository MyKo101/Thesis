% This is the Reed College LaTeX thesis template. Most of the work
% for the document class was done by Sam Noble (SN), as well as this
% template. Later comments etc. by Ben Salzberg (BTS). Additional
% restructuring and APA support by Jess Youngberg (JY).
% Your comments and suggestions are more than welcome; please email
% them to cus@reed.edu
%
% See http://web.reed.edu/cis/help/latex.html for help. There are a
% great bunch of help pages there, with notes on
% getting started, bibtex, etc. Go there and read it if you're not
% already familiar with LaTeX.
%
% Any line that starts with a percent symbol is a comment.
% They won't show up in the document, and are useful for notes
% to yourself and explaining commands.
% Commenting also removes a line from the document;
% very handy for troubleshooting problems. -BTS

% As far as I know, this follows the requirements laid out in
% the 2002-2003 Senior Handbook. Ask a librarian to check the
% document before binding. -SN

%%
%% Preamble
%%
% \documentclass{<something>} must begin each LaTeX document
\documentclass[12pt,twoside]{reedthesis}
% Packages are extensions to the basic LaTeX functions. Whatever you
% want to typeset, there is probably a package out there for it.
% Chemistry (chemtex), screenplays, you name it.
% Check out CTAN to see: http://www.ctan.org/
%%
\usepackage{graphicx,latexsym}
\usepackage{amsmath}
\usepackage{amssymb,amsthm}
\usepackage{longtable,booktabs,setspace}
\usepackage{chemarr} %% Useful for one reaction arrow, useless if you're not a chem major
\usepackage[hyphens]{url}
% Added by CII
\usepackage{hyperref}
\usepackage{lmodern}
\usepackage{float}
\floatplacement{figure}{H}
% End of CII addition
\usepackage{rotating}

% Next line commented out by CII
%%% \usepackage{natbib}
% Comment out the natbib line above and uncomment the following two lines to use the new
% biblatex-chicago style, for Chicago A. Also make some changes at the end where the
% bibliography is included.
%\usepackage{biblatex-chicago}
%\bibliography{thesis}


% Added by CII (Thanks, Hadley!)
% Use ref for internal links
\renewcommand{\hyperref}[2][???]{\autoref{#1}}
\def\chapterautorefname{Chapter}
\def\sectionautorefname{Section}
\def\subsectionautorefname{Subsection}
% End of CII addition

% Added by CII
\usepackage{caption}
\captionsetup{width=5in}
% End of CII addition

% \usepackage{times} % other fonts are available like times, bookman, charter, palatino

% Syntax highlighting #22

% To pass between YAML and LaTeX the dollar signs are added by CII
\title{Multi-State Clinical Prediction Models in Renal Replacement Therapy}
\author{Michael Andrew Barrowman}
% The month and year that you submit your FINAL draft TO THE LIBRARY (May or December)
\date{March 2020}
\division{Division of Informatics, Imaging and Data Science}
\advisor{Dr.~Matthew Sperrin, Prof.~Niels Peek, Dr.~Glen Martin, Dr.~Mark Lambie}
\institution{University of Manchester}
\degree{PhD Medicine}
%If you have two advisors for some reason, you can use the following
% Uncommented out by CII
% End of CII addition

%%% Remember to use the correct department!
\department{School of Health Sciences}
% if you're writing a thesis in an interdisciplinary major,
% uncomment the line below and change the text as appropriate.
% check the Senior Handbook if unsure.
%\thedivisionof{The Established Interdisciplinary Committee for}
% if you want the approval page to say "Approved for the Committee",
% uncomment the next line
%\approvedforthe{Committee}

\newlength{\cslhangindent}
\setlength{\cslhangindent}{1.5em}
\newenvironment{cslreferences}%
  {\setlength{\parindent}{0pt}%
  \everypar{\setlength{\hangindent}{\cslhangindent}}\ignorespaces}%
  {\par}

% Added by CII
%%% Copied from knitr
%% maxwidth is the original width if it's less than linewidth
%% otherwise use linewidth (to make sure the graphics do not exceed the margin)
\makeatletter
\def\maxwidth{ %
  \ifdim\Gin@nat@width>\linewidth
    \linewidth
  \else
    \Gin@nat@width
  \fi
}
\makeatother

\renewcommand{\contentsname}{Table of Contents}
% End of CII addition

\setlength{\parskip}{0pt}

% Added by CII

\providecommand{\tightlist}{%
  \setlength{\itemsep}{0pt}\setlength{\parskip}{0pt}}

\Acknowledgements{
I would like to thank my\ldots{}
}

\Dedication{

}

\Preface{

}

\Abstract{
The preface pretty much says it all.

\par

Second paragraph of abstract starts here.
}

% End of CII addition
%%
%% End Preamble
%%
%
\begin{document}

% Everything below added by CII
  \maketitle

\frontmatter % this stuff will be roman-numbered
\pagestyle{empty} % this removes page numbers from the frontmatter
  \begin{acknowledgements}
    I would like to thank my\ldots{}
  \end{acknowledgements}

  \hypersetup{linkcolor=black}
  \setcounter{tocdepth}{2}
  \tableofcontents

  \listoftables

  \listoffigures
  \begin{abstract}
    The preface pretty much says it all.

    \par

    Second paragraph of abstract starts here.
  \end{abstract}

\mainmatter % here the regular arabic numbering starts
\pagestyle{fancyplain} % turns page numbering back on

\hypertarget{literature-review}{%
\chapter{Literature Review}\label{literature-review}}

This is the first chapter of my thesis and will include a brief summary of what the current literature looks like. It will be split into sections and subsections as specified in my ToDo List

\hypertarget{introduction}{%
\section{Introduction}\label{introduction}}

lorem ipsum blah blah blah

\hypertarget{clinical-prediction-models}{%
\section{Clinical Prediction Models}\label{clinical-prediction-models}}

The idea of prognosis dates back to ancient Greece with the work of Hippocrates {[}\textbf{Cite: Hippocrates}{]} and is derived from the Greek for ``know before'' meaning to forecast the future. Within the sphere of healthcare\ldots{}

Prognosis research can be broken down into four main categories (with three subcategories {[}\textbf{Cite: Riley\_Book}{]}):
\begin{itemize}
\tightlist
\item
  Type I: Fundamental prognosis research {[}\textbf{Cite: LR:1}{]}
\item
  Type II: Prognostic factor research{[}\textbf{Cite: LR:2}{]}
\item
  Type III: Prognostic model research{[}\textbf{Cite: LR:3}{]}
  \begin{itemize}
  \tightlist
  \item
    Model development{[}\textbf{Cite: LR:34}{]}
  \item
    Model validation{[}\textbf{Cite: LR:35}{]}
  \item
    Model impact evaluation{[}\textbf{Cite: LR:36}{]}
  \end{itemize}
\item
  Type IV: Stratified Medicine {[}\textbf{Cite: LR:4}{]}
\end{itemize}
\hypertarget{fundamental-prognosis-research}{%
\subsection{Fundamental Prognosis Research}\label{fundamental-prognosis-research}}

What is it?

\hypertarget{prognostic-factor-research}{%
\subsection{Prognostic Factor Research}\label{prognostic-factor-research}}

The aim of prognostic factor research (Type II) is to discover which factors are associated with disease progression. This allows for the general attribution of relationships between predictors and clinical outcomes.

Predictive factor research can give researchers and clinicians an idea of which patient factors are important when assessing a disease. It is vital to the development of clinical predictive models as without an idea of what covariates \emph{can} affect an outcome, we cannot figure out which variables \emph{will} affect the outcome. For example, {[}\textbf{xxxx}{]} demonstrated that {[}\textbf{xxxx}{]} is correlated with {[}\textbf{xxxx}{]}, which subsequently used as a covariate in the development of the {[}\textbf{xxxx}{]} model.

\hypertarget{prognostic-model-research}{%
\subsection{Prognostic Model Research}\label{prognostic-model-research}}

\hypertarget{model-development}{%
\subsubsection{Model Development}\label{model-development}}

\hypertarget{model-validation}{%
\subsubsection{Model Validation}\label{model-validation}}

\hypertarget{impact-evaluation}{%
\subsubsection{Impact Evaluation}\label{impact-evaluation}}

\hypertarget{stratified-medicine}{%
\subsection{Stratified Medicine}\label{stratified-medicine}}

\hypertarget{examples}{%
\subsection{Examples}\label{examples}}

\hypertarget{competing-risks-multi-state-models}{%
\section{Competing Risks \& Multi-State Models}\label{competing-risks-multi-state-models}}

\hypertarget{math-sci}{%
\chapter{Mathematics and Science}\label{math-sci}}

\hypertarget{math}{%
\section{Math}\label{math}}

\TeX~is the best way to typeset mathematics. Donald Knuth designed \TeX~when he got frustrated at how long it was taking the typesetters to finish his book, which contained a lot of mathematics. One nice feature of \emph{R Markdown} is its ability to read LaTeX code directly.

If you are doing a thesis that will involve lots of math, you will want to read the following section which has been commented out. If you're not going to use math, skip over or delete this next commented section.

\hypertarget{chemistry-101-symbols}{%
\section{Chemistry 101: Symbols}\label{chemistry-101-symbols}}

Chemical formulas will look best if they are not italicized. Get around math mode's automatic italicizing in LaTeX by using the argument \texttt{\$\textbackslash{}mathrm\{formula\ here\}\$}, with your formula inside the curly brackets. (Notice the use of the backticks here which enclose text that acts as code.)

So, \(\mathrm{Fe_2^{2+}Cr_2O_4}\) is written \texttt{\$\textbackslash{}mathrm\{Fe\_2\^{}\{2+\}Cr\_2O\_4\}\$}.

\noindent Exponent or Superscript: \(\mathrm{O^-}\)

\noindent Subscript: \(\mathrm{CH_4}\)

To stack numbers or letters as in \(\mathrm{Fe_2^{2+}}\), the subscript is defined first, and then the superscript is defined.

\noindent Bullet: CuCl \(\bullet\) \(\mathrm{7H_{2}O}\)

\noindent Delta: \(\Delta\)

\noindent Reaction Arrows: \(\longrightarrow\) or \(\xrightarrow{solution}\)

\noindent Resonance Arrows: \(\leftrightarrow\)

\noindent Reversible Reaction Arrows: \(\rightleftharpoons\)

\hypertarget{typesetting-reactions}{%
\subsection{Typesetting reactions}\label{typesetting-reactions}}

You may wish to put your reaction in an equation environment, which means that LaTeX will place the reaction where it fits and will number the equations for you.
\begin{equation}
  \mathrm{C_6H_{12}O_6  + 6O_2} \longrightarrow \mathrm{6CO_2 + 6H_2O}
  \label{eq:reaction}
\end{equation}
We can reference this combustion of glucose reaction via Equation \eqref{eq:reaction}.

\hypertarget{other-examples-of-reactions}{%
\subsection{Other examples of reactions}\label{other-examples-of-reactions}}

\(\mathrm{NH_4Cl_{(s)}}\) \(\rightleftharpoons\) \(\mathrm{NH_{3(g)}+HCl_{(g)}}\)

\noindent \(\mathrm{MeCH_2Br + Mg}\) \(\xrightarrow[below]{above}\) \(\mathrm{MeCH_2\bullet Mg \bullet Br}\)

\hypertarget{physics}{%
\section{Physics}\label{physics}}

Many of the symbols you will need can be found on the math page \url{http://web.reed.edu/cis/help/latex/math.html} and the Comprehensive LaTeX Symbol Guide (\url{http://mirror.utexas.edu/ctan/info/symbols/comprehensive/symbols-letter.pdf}).

\hypertarget{biology}{%
\section{Biology}\label{biology}}

You will probably find the resources at \url{http://www.lecb.ncifcrf.gov/~toms/latex.html} helpful, particularly the links to bsts for various journals. You may also be interested in TeXShade for nucleotide typesetting (\url{http://homepages.uni-tuebingen.de/beitz/txe.html}). Be sure to read the proceeding chapter on graphics and tables.

\backmatter

\hypertarget{references}{%
\chapter*{References}\label{references}}
\addcontentsline{toc}{chapter}{References}

\markboth{References}{References}

\noindent

\setlength{\parindent}{-0.20in}
\setlength{\leftskip}{0.20in}
\setlength{\parskip}{8pt}

\hypertarget{refs}{}
\begin{cslreferences}
\leavevmode\hypertarget{ref-angel2000}{}%
Angel, E. (2000). \emph{Interactive computer graphics : A top-down approach with opengl}. Boston, MA: Addison Wesley Longman.

\leavevmode\hypertarget{ref-angel2001}{}%
Angel, E. (2001a). \emph{Batch-file computer graphics : A bottom-up approach with quicktime}. Boston, MA: Wesley Addison Longman.

\leavevmode\hypertarget{ref-angel2002a}{}%
Angel, E. (2001b). \emph{Test second book by angel}. Boston, MA: Wesley Addison Longman.
\end{cslreferences}

% Index?

\end{document}
