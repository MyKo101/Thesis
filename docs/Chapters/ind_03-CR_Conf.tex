% Options for packages loaded elsewhere
\PassOptionsToPackage{unicode}{hyperref}
\PassOptionsToPackage{hyphens}{url}
%
\documentclass[
]{article}
\usepackage{lmodern}
\usepackage{amssymb,amsmath}
\usepackage{ifxetex,ifluatex}
\ifnum 0\ifxetex 1\fi\ifluatex 1\fi=0 % if pdftex
  \usepackage[T1]{fontenc}
  \usepackage[utf8]{inputenc}
  \usepackage{textcomp} % provide euro and other symbols
\else % if luatex or xetex
  \usepackage{unicode-math}
  \defaultfontfeatures{Scale=MatchLowercase}
  \defaultfontfeatures[\rmfamily]{Ligatures=TeX,Scale=1}
\fi
% Use upquote if available, for straight quotes in verbatim environments
\IfFileExists{upquote.sty}{\usepackage{upquote}}{}
\IfFileExists{microtype.sty}{% use microtype if available
  \usepackage[]{microtype}
  \UseMicrotypeSet[protrusion]{basicmath} % disable protrusion for tt fonts
}{}
\makeatletter
\@ifundefined{KOMAClassName}{% if non-KOMA class
  \IfFileExists{parskip.sty}{%
    \usepackage{parskip}
  }{% else
    \setlength{\parindent}{0pt}
    \setlength{\parskip}{6pt plus 2pt minus 1pt}}
}{% if KOMA class
  \KOMAoptions{parskip=half}}
\makeatother
\usepackage{xcolor}
\IfFileExists{xurl.sty}{\usepackage{xurl}}{} % add URL line breaks if available
\IfFileExists{bookmark.sty}{\usepackage{bookmark}}{\usepackage{hyperref}}
\hypersetup{
  pdftitle={How unmeasured confounding in a competing risks setting can affect treatment effect estimates in observational studies},
  pdfauthor={MA Barrowman; N Peek; M Lambie; GP Martin; M Sperrin},
  hidelinks,
  pdfcreator={LaTeX via pandoc}}
\urlstyle{same} % disable monospaced font for URLs
\usepackage[margin=1in]{geometry}
\usepackage{longtable,booktabs}
% Correct order of tables after \paragraph or \subparagraph
\usepackage{etoolbox}
\makeatletter
\patchcmd\longtable{\par}{\if@noskipsec\mbox{}\fi\par}{}{}
\makeatother
% Allow footnotes in longtable head/foot
\IfFileExists{footnotehyper.sty}{\usepackage{footnotehyper}}{\usepackage{footnote}}
\makesavenoteenv{longtable}
\usepackage{graphicx}
\makeatletter
\def\maxwidth{\ifdim\Gin@nat@width>\linewidth\linewidth\else\Gin@nat@width\fi}
\def\maxheight{\ifdim\Gin@nat@height>\textheight\textheight\else\Gin@nat@height\fi}
\makeatother
% Scale images if necessary, so that they will not overflow the page
% margins by default, and it is still possible to overwrite the defaults
% using explicit options in \includegraphics[width, height, ...]{}
\setkeys{Gin}{width=\maxwidth,height=\maxheight,keepaspectratio}
% Set default figure placement to htbp
\makeatletter
\def\fps@figure{htbp}
\makeatother
\setlength{\emergencystretch}{3em} % prevent overfull lines
\providecommand{\tightlist}{%
  \setlength{\itemsep}{0pt}\setlength{\parskip}{0pt}}
\setcounter{secnumdepth}{5}
\newcommand{\txt}[1]{\textrm{#1}}

\def\logit{\txt{logit}}

\newcommand{\sfrac}[2]{\;^{#1}/_{#2}}
\usepackage{amsmath}
\usepackage[T1]{fontenc}
\usepackage{rotating}
\usepackage{booktabs}
\usepackage{longtable}
\usepackage{array}
\usepackage{multirow}
\usepackage{wrapfig}
\usepackage{float}
\usepackage{colortbl}
\usepackage{pdflscape}
\usepackage{tabu}
\usepackage{threeparttable}
\usepackage{threeparttablex}
\usepackage[normalem]{ulem}
\usepackage{makecell}
\usepackage{xcolor}

\title{How unmeasured confounding in a competing risks setting can affect treatment effect estimates in observational studies}
\author{MA Barrowman \and N Peek \and M Lambie \and GP Martin \and M Sperrin}
\date{}
\begin{document}
\maketitle

{
\setcounter{tocdepth}{2}
\tableofcontents
}
Published as: \textbf{MA Barrowman}, N Peek, M Lambie et al, How unmeasured confounding in a competing risks setting can affect treatment effect estimates in observational studies, BMC Medical Research Methodology (2019) doi: \href{https://doi.org/10.1186/s12874-019-0808-7}{10.1186/s12874-019-0808-7}

\hypertarget{abstract}{%
\section*{Abstract}\label{abstract}}
\addcontentsline{toc}{section}{Abstract}

\hypertarget{background}{%
\subsection{Background}\label{background}}

Analysis of competing risks is commonly achieved through a cause specific or a subdistribution framework using Cox or Fine \& Gray models, respectively. The estimation of treatment effects in observational data is prone to unmeasured confounding which causes bias. There has been limited research into such biases in a competing risks framework.

\hypertarget{methods}{%
\subsection{Methods}\label{methods}}

We designed simulations to examine bias in the estimated treatment effect under Cox and Fine \& Gray models with unmeasured confounding present. We varied the strength of the unmeasured confounding (i.e.~the unmeasured variable's effect on the probability of treatment and both outcome events) in different scenarios.

\hypertarget{results}{%
\subsection{Results}\label{results}}

In both the Cox and Fine \& Gray models, correlation between the unmeasured confounder and the probability of treatment created biases in the same direction (upward/downward) as the effect of the unmeasured confounder on the event-of-interest. The association between correlation and bias is reversed if the unmeasured confounder affects the competing event. These effects are reversed for the bias on the treatment effect of the competing event and are amplified when there are uneven treatment arms.

\hypertarget{conclusion}{%
\subsection{Conclusion}\label{conclusion}}

The effect of unmeasured confounding on an event-of-interest or a competing event should not be overlooked in observational studies as strong correlations can lead to bias in treatment effect estimates and therefore cause inaccurate results to lead to false conclusions. This is true for cause specific perspective, but moreso for a subdistribution perspective. This can have ramifications if real-world treatment decisions rely on conclusions from these biased results. Graphical visualisation to aid in understanding the systems involved and potential confounders/events leading to sensitivity analyses that assumes unmeasured confounders exists should be performed to assess the robustness of results.

\hypertarget{background-1}{%
\section{Background}\label{background-1}}

Well-designed observation studies permit researchers to assess treatment effects when randomisation is not feasible. This may be due to cost, suspected non-equipoise treatments or any number of other reasons {[}1{]}. While observational studies minimise these issues by being cheaper to run and avoiding randomisation (which, although unknown at the time, may prescribe patients to worse treatments), they are potentially subject to issues such as unmeasured confounding and increased possibility of competing risks (where multiple clinically relevant events occur). Although these issues can arise in any study, Randomised Controlled Trials (RCTs) attempt to mitigate these effects by using randomisation of treatment and strict inclusion/exclusion criteria. However, the estimated treatment effects from RCTs are of potentially limited generalisability, accessibility and implementability {[}2{]}.

A confounder is a variable that is a common cause of both treatment and outcome. For example, a patient with a high Body Mass Index (BMI) is more likely to be prescribed statins {[}3{]}, but are also more likely to suffer a cardiovascular event. These treatment decisions can be affected by variables that are not routinely collected (such as childhood socio-economic status or the severity of a comorbidity {[}4{]}. Therefore, if these variables are omitted form (or unavailable for) the analysis of treatment effects in observational studies, then they can bias inferences {[}5{]}. As well as having a direct effect on the event-of-interest, confounders (along with other covariates) can also have further reaching effects on a patient's health by changing the chances of having a competing event. Patients who are more likely to have a competing event are less likely to have an event-of-interest, which can affect inferences from studies ignoring the competing event. In the above BMI example, a high BMI can also increase a patient's likelihood of developing (and thus dying from) cancer {[}6{]}.

The issue of confounding in observational studies has been researched previously {[}7,8,9{]}, where it has been consistently shown that unmeasured confounding is likely to occur within these natural datasets and that there is poor reporting of this, even after the introduction of the The Strengthening the Reporting of Observational Studies in Epidemiology (STROBE) Guidelines {[}10, 11{]}. Hence, it is widely recognised that sensitivity analyses are vital within the observational setting {[}12{]}. However these previous studies do not extend this work into a competing risk setting, meaning research in this space is lacking {[}13{]}, particularly where the presence of a competing event can affect the rate of occurrence of the event-of-interest. These issues will commonly occur in elderly and comorbid patients where treatment decisions are more complex. As the elderly population grows, the clinical community needs to understand the optimal way to treat patients with complex conditions; here, causal relationships between treatment and outcome need to account for competing events appropriately.

The most common way of analysing data that contains competing events is using a cause specific perspective, as in the Cox methodology {[}14{]}, where competing events are considered as censoring events and analysis focuses solely on the event-of-interest. The alternative is to assume a subdistributional perspective, as in the Fine \& Gray methodology {[}15{]}, where patients who have competing events remain in the risk set forever.

The aim of this paper is to study the bias induced by the presence of unmeasured confounding on treatment effect estimates in the competing risks framework. We investigated how unmeasured confounding affects the apparent effect of treatment under the Fine \& Gray and the Cox methodologies and how these estimates differ from their true value. To accomplish this, we used simulations to generate synthetic time-to-event-data and then model under both perspectives. Both the Cox and Fine \& Gray models provide hazard ratios to describe the effects of a covariate. A binary covariate will represent a treatment and the coefficients found by the model will be the estimate of interest.

\hypertarget{methods-1}{%
\section{Methods}\label{methods-1}}

We considered a simulation scenario in which our population can experience two events; one of which is the event-of-interest (Event 1), the other is a competing event (Event 2). We model a single unmeasured confounding covariate, \(U \sim N (0,1)\) and a binary treatment indicator, \(Z\). We varied how much \(U\) and \(Z\) affect the probability distribution of the two events as well as how they are correlated. For example, \(Z\) could represent whether a patient is prescribed statins, U could be their BMI, the event-of-interest could be cardiovascular disease related mortality and a competing event could be cancer-related mortality. We followed best practice for conducting and reporting simulations studies {[}16{]}.

The data-generating mechanism defined two cause-specific hazard functions (one for each event), where the baseline hazard for event 1 was \(k\) times that of event 2, see Fig. 1. We assumed a baseline hazard that was either constant (exponential distributed failure times), linearly increasing (Weibull distributed failure times) or biologically plausible {[}17{]}. The hazards used were thus:
\begin{align}
\lambda_1(t|U,Z) &= ke^{\beta_1U + \gamma_1Z}\lambda_0(t)\\
\lambda_2(t|U,Z) &= ke^{\beta_2U + \gamma_2Z}\lambda_0(t)
\end{align}
\begin{equation}
\lambda_0(t) \begin{cases}
1 & \textrm{Exponential}\\
2t & \textrm{Webull}\\
\exp{-18+7.3t-11.5t^{0.5}\log(t) + 9.5t^{0.5}} & \textrm{Plausible}
\end{cases}
\end{equation}
In the above equations, \(\beta\) and \(\gamma\) are the effects of the confounding covariate and the treatment effect respectively with the subscripts representing which event they are affecting. These two hazard functions entirely describe how a population will behave {[}18{]}.

{[}\textbf{Insert Figure 1}{]}

We simulated populations of 10,000 patients to ensure small confidence intervals around our treatment effect estimates in each simulation. Each simulated population had a distinct value for \(\beta\) and \(\gamma\). In order to simulate the confounding of \(U\) and \(Z\), we generated these values such that \(\textrm{Corr}(U,Z) = \rho\) and \(\Pr(Z = 1) = \pi\) {[}19{]}. Population end times and type of event were generated using the relevant hazard functions. The full process for the simulations can be found in Additional file 1. Due to the methods used to generate the populations, the possible values for \(\rho\) are bounded by the choice of \(\pi\) such that when \(\pi = 0.5\), \(\left|\rho\right| <= 0.797\) and when \(\pi = 0.1\) (or \(\pi=0.9\)), \(\left|\rho\right| <= 0.57\). The relationship between the parameters can be seen in the Directed Acyclic Graph (DAG) shown in Fig. 2, where \(T\) is the event time and \(\delta\) is the event type indicator (1 for event-of-interest and 2 for competing event).
\# (APPENDIX) Appendices \{-\}

\hypertarget{chapConfCRsupp}{%
\section{Supplementary Material}\label{chapConfCRsupp}}

\hypertarget{simulation-details}{%
\subsection{Simulation Details}\label{simulation-details}}

The populations

\hypertarget{mathematics-of-subdistribution-hazards}{%
\subsection{Mathematics of Subdistribution Hazards}\label{mathematics-of-subdistribution-hazards}}

Due to the relationship between the cause specific hazard functions and the subdistribution hazard functions they cannot both satisfy the proportional hazards assumption. We have defined CSH functions to be proportio

\end{document}
